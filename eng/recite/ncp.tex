\chapter{New Concept English}
\begin{itemize}
\item The boy put on his goggles, fitted them tight, tested the vacuum. His hands were shaking. Then he chose the biggest stone he could carry and slipped over the edge of the rock until half of him was in the cool, enclosing water and half in the hot sun. He looked up once at the empty sky, filled his lungs once, twice, and then sank fast to the bottom with the stone. He let it go and began to count. He took the edges of the hole in his hands and drew himself into it, wriggling his shoulders in sideways as he remembered he must, kicking himself along with his feet. Soon he was clear inside. He was in a small rock-bound hole filled with yellowish-grey water. The water was pushing him up against the root. The roof was sharp and pained his back. He pulled himself along with his hands - fast, fast - and used his legs as levers. His head knocked against something; a sharp pain dizzied him. Fifty, fifty-one, fifty-two ... He was without light, and the water seemed to press upon him with the weight of rock. Seventy-one, seventy-two ... There was no strain on his lungs. He felt like an inflated balloon, his lungs were so light and easy, but his head was pulsing. He was being continually pressed against the sharp roof, which felt slimy as well as sharp. Again he thought of octopuses, and wondered if the tunnel might be filled with weed that could tangle him. He gave himself a panicky, convulsive kick forward, ducked his head, and swam. His feet and hands moved freely, as if in open water. The hole must have widened out. He thought he must be swimming fast, and he was frightened of banging his head if the tunnel narrowed. A hundred, a hundred and one ... The water paled. Victory filled him. His lungs were beginning to hurt. A few more strokes and he would be out. He was counting wildly; he said a hundred and fifteen, and then, a long time later, a hundred and fifteen again. The water was a clear jewel-green all around him. Then he saw, above his head, a crack running up through the rock. Sunlight was falling through it, showing the clean dark rock of the tunnel, a single mussel shell and darkness ahead. He was at the end of what he could do. He looked up at the crack as if it were filled with air and not water, as if he could up his mouth to it to draw in air. A hundred and fifteen, he heard himself say inside his head -  but he had said that long ago. He must go on into the blackness ahead, or he would drown. His head was swelling, his lungs cracking. A hundred and fifteen, a hundred and fifteen pounded through his head, and he feebly clutched at rocks in the dark, pulling himself forward, leaving the brief space of sunlit water behind. He felt he was dying. He was no longer quite conscious. He struggled on in the darkness cracked with an explosion of green light. His hands, groping forward, met nothing, and his feet, kicking back, propelled him out into the open sea.


\item Television is a method of communication. It is about as revolutionary as the invention of printing. Neither printing nor television is in itself an idea, or power, or good or bad. They are simply methods by which ideas and experiences can be communicated faster to more people. It is perhaps becaue the characteristics of television, which determine what it can best communicate, are so different from those of printing, that professional educationists were reluctant for so long to interest themselves in the newer method.
Printing and television are certainly alike in that both are costly to the producers of the communication and relatively cheap to the receiver. They are both, therefore, mass media which depend upon reaching great numbers. But whereas the printed word, being relatively permanent, can communicate to numbers of like minds over centuries, television is relatively ephemeral and communicates, using both pictures and words, to millions of unlike minds at the same moment in time. Moreover television appeals not only to those who can read but to those who can't. Professional educationists, accustomed to communication through words, and highly valuing reading and the quality of the like minds reachable through books, saw television, in its early years, not only as a rival for attention but as an enemy of the good. Some ten years ago a friend said to me: 'We in Oxford may be old fashioned and fuddy-duddy, but most of us think that television is actively detrimental.' Even that great pioneer of teaching by radio, the late Mary Somerville, had no faith in television. 'It won't last,' she said to me. 'It's a flash in the pan.' And many in the world of education no dbout hoped that this was true. 
The situation has now altered. It is clear that televisions is no flash in the pan. So educationists all over the world are trying to get access to its 'power', often by attempting to use traditional methods of academic teaching to inculcate, through television, the ideas and attitudes in which they devoutedly believe. But one of the characteristics of television is that it has no power other than that created by the wish of people to watch it. If nobody watches it, then television has no power.
\end{itemize}
\begin{enumerate}

\item    Finding fossil man 

Why are legends handed down by storytellers useful?  We can read of things that happened  years ago in the Near East where people first learned to write.
But there are some parts of the word where even now people cannot write.
The only way that they can preserve their history is to recount it as sagas - legends handed down from one generation of another.
These legends are useful because they can tell us something about migrations of people who lived long ago but none could write down what they did.
Anthropologists wondered where the remote ancestors of the Polynesian peoples now living in the Pacific Islands came from.
The sagas of these people explain that some of them came from Indonesia about  years ago.
But the first people who were like ourselves lived so long ago that even their sagas if they had any are forgotten.
So archaeologists have neither history nor legends to help them to find out where the first modern men came from.
Fortunately however ancient men made tools of stone especially flint because this is easier to shape than other kinds.
They may also have used wood and skins but these have rotted away.
Stone does not decay and so the tools of long ago have remained when even the bones of the men who made them have disappeared without trace.
\item Spare that spider

How much of each year do spiders spend killing insects?  Why you may wonder should spiders be our friends?   Because they destroy so many insects and insects include some of the greatest enemies of the human race.
Insects would make it impossible for us to live in the world they would devour all our crops and kill our flocks and herds if it were not for the protection we get from insect-eating animals.
We owe a lot to the birds and beasts who eat insects but all of them put together kill only a fraction of the number destroyed by spiders.
Moreover unlike some of the other insect eaters spiders never do the harm to us or our belongings.
Spiders are not insects as many people think nor even nearly related to them.
One can tell the difference almost at a glance for a spider always has eight legs and insect never more than six.
How many spiders are engaged in this work no our behalf?   One authority on spiders made a census of the spiders in grass field in the south of England and he estimated that there were more than 2,225,000 in one acre that is something like 6,000,000 spiders of different kinds on a football pitch.
Spiders are busy for at least half the year in killing insects.
It is impossible to make more than the wildest guess at how many they kill but they are hungry creatures not content with only three meals a day.
It has been estimated that the weight of all the insects destroyed by spiders in Britain in one year would be greater than the total weight of all the human beings in the country.
                
\item Matterhorn man
                  
What was the main objective of early mountain climbers?  Modern alpinists try to climb mountains by a route which will give them good sport and the more difficult it is the more highly it is regarded.
In the pioneering days however this was not the case at all.
The early climbers were looking for the easiest way to the top because the summit was the prize they sought, especially if it and never been attained before.
It is true that during their explorations they often faced difficulties and dangers of the most perilous nature equipped in a manner with would make a modern climber shudder at the thought but they did not go out of their way to court such excitement.
They had a single aim a solitary goal  the topIt is hard for us to realize nowadays how difficult it was for the pioneers.
Except for one or two places such as Zermatt and Chamonix which had rapidly become popular, Alpine village tended to be impoverished settlements cut off from civilization by the high mountains.
Such inns as there were generally dirty and flea-ridden the food simply local cheese accompanied by bread often twelve months old all washed down with coarse wine.
Often a valley boasted no inn at all and climbers found shelter wherever they could - sometimes with the local priest (who was usually as poor as his parishioners) sometimes with shepherds or cheesemakers.
Invariably the background was the same dirt and poverty and very uncomfortable.
For men accustomed to eating seven-course dinners and sleeping between fine linen sheets at home, the change to the Alps must have very hard indeed.

\item  Seeing hands

How did Vera discover she had this gift of second sight? Several cases have been reported in Russia recently of people who can detect colors with their fingers and even see through solid and walls.
One case concerns and eleven-year-old schoolgirl Vera Petrova who has normal vision but who can also perceive things with different parts of her skin and through solid walls.
This ability was first noticed by her father.
One day she came into his office and happened to put her hands on the door of a locked safe.
Suddenly she asked her father why he kept so many old newspapers locked away there and even described the way they were done up in bundles.
Vera's curious talent was brought to the notice of a scientific research institute in the town of Ulyanovsk near where she lives, and in April she was given a series of tests by a special commission of the Ministry of Health of the Russian Federal Republic.
During these tests she was able to read a newspaper through an opaque screen, and stranger, still by moving her elbow over a child's game of Lotto she was able to describe the figures and colors printed on it and in another instance wearing stockings and slippers to make out with her foot the outlines and colors of a picture hidden under a carpet.
Other experiments showed that her knees and shoulders had a similar sensitivity.
During all these tests Vera was blindfold and indeed except when blindfold she lacked the ability to perceive things with her skin.
It was also found that although she could perceive things with her fingers this ability ceased the moment her hands were wet.


\item Youth 

How does the writer like to treat young people? People are always talking about the problem of youth.
If there is one - which I take leave to doubt  then it is older people who create it not the young themselves.
Let us get down to fundamentals and agree that the young are after all human beings - people just like their elders.
There is only one difference between an old man and a young one: the young man has a glorious future before him and the old one has a splendid future behind him and maybe that is where the rub is.
When I was a teenager I felt that I was just young and uncertain - that I was a new boy in a huge school and I would have been very pleased to be regarded as something so interesting as a problem.
For one thing being a problem gives you a certain identity and that is one of the things the young are busily engaged in seeking.
I find young people exciting.
They have an air of freedom and they not a dreary commitment to mean ambitions or love of comfort.
They are not anxious social climbers and they have no devotion to material things.
All this seems to me to link them with life and the origins of things.
Its as if they were in some sense cosmic beings in violent and lovely contrast with us suburban creatures.
All that is in my mind when I meet a young person.
He may be conceited ill-mannered presumptuous or fatuous but I do not turn for protection to dreary cliches about respect of elders - as if mere age were a reason for respect.
I accept that we are equals and I will argue with him as an equal if I think he is wrong.


\item The sporting spirit

How does the writer describe sport at the international level? I am always amazed when I hear people saying that sport creates goodwill between the nations and that if only the common peoples of the would could meet one another at football or cricket they would have no inclination to meet on the battlefield.
Even if one didn't know from concrete examples (the 1936 Olympic Games for instance) that international sporting contests lead to orgies of hatred, one could deduce if from general principles.
Nearly all the sports practiced nowadays are competitive.
You play to win and the game has little meaning unless you do your utmost to win.
On the village green, where you pick up sides and no feeling of local patriotism is involved, it is possible to play simply for the fun and exercise: but as soon as a the question of prestige arises, as soon as you feel that you and some larger unit will be disgraced if you lose, the most savage combative instincts are aroused.
Anyone who has played even in a school football match knows this.
At the international level, sport is frankly mimic warfare.
But the significant thing is not the behaviour of the players but the attitude of the spectators: and, behind the spectators, of the nations who work themselves into furies over these absurd contests and seriously believe - at any rate for short periods - that running, jumping and kicking a ball are tests of national virtue.

\item  Bats

In what way does echolocation in bats play an utilitarian role? Not all sounds made by animals serve as language and we have only to turn to that extraordinary discovery of echo-location in bats to see a case in which the voice plays a strictly utilitarian role.
To get a full appreciation of what this means we must turn first to some recent human inventions.
Everyone knows that if he shouts in the vicinity of a wall or a mountainside an echo will come back.
The further off this solid obstruction the longer time will elapse for the return of the echo.
A sound made by tapping on the hull of a ship will be reflected from the sea bottom, and by measuring the time interval between the taps and the receipt of the echoes the depth of the sea at that point can be calculated.
So was born the echo-sounding apparatus now in general use in ships.
Every solid object will reflect a sound varying according to the size and nature of the object.
A shoal of fish will do this.
So it is a comparatively simple step from locating the sea bottom to locating a shoal of fish.
With experience and with improved apparatus it is now possible not only to locate a shoal but to tell if it is herring cod or other well-known fish by the pattern of its echo.
It has been found that certain bats emit squeaks and by receiving the echoes they can locate and steer clear of obstacles  - or locate flying insects on which they feed.
This echolocation in bats is often compared with radar the principle of which is similar.


\item Trading standards

What makes trading between rich countries difficult? Chickens slaughtered in the United States, claim officials in Brussels, are not fit to grace European tables.
No, say the American, our fowl are fine we simply clean them in a different way.
These days, it is differences in national regulations, far more than tariffs, that put sand in the wheels of trade between rich countries.
It is not just farmers who are complaining.
An electric razor that meets the European Unions safety standards must be approved by American testers before it can be sold in the United States, and an American-made dialysis machine needs the EU's okay before is hits the market in Europe.
As it happens, a razor that is safe in Europe is unlikely to electrocute Americans.
So, ask businesses on both sides of the Atlantic, why have two lots of tests where one would do? Politicians agree, in principle, so America and the EU have been trying to reach a deal which would eliminate the need to double-test many products.
They hope to finish in time for a trade summit between America and the EU on May 28th.
Although negotiators are optimistic the details are complex enough that they may be hard-pressed to get a deal at all.
Why? One difficulty is to construct the agreements.
The Americans would happily reach one accord on standards for medical devices and them hammer out different pacts covering, say, electronic goods and drug manufacturing.
The EU - following fine continental traditions - wants agreement on general principles which could be applied to many types of products and perhaps extended to other countries.
\item  Royal espionage

What important thing did King Alfred learn when he penetrated the Danish camp of Guthrum? 
Alfred the Great acted his own spy visiting Danish camps disguised as a minstrel.
In those days wandering minstrels were welcome everywhere.
They were not fighting men and their harp was their passport.
Alfred had learned many of their ballads in his youth and could vary his program with acrobatic tricks and simple conjuring.
While Alfred's little army slowly began to gather at Athelney, the king himself set out to penetrate the camp of Guthrum, the commander of the Danish invaders.
There had settled down for the winter at Chippenham: thither Alfred went.
He noticed at once that discipline was slack: the Danes had the self-confidence of conquerors and their security precautions were casual.
They lived well on the proceeds of raids on neighboring regions.
There they collected women as well as food and drink and a life of ease had made them soft.
Alfred stayed in the camp a week before he returned to Athelney.
The force there assembled was trivial compared with the Danish horde.
But Alfred had deduced that the Danes were no longer fit for prolonged battle: and that their commissariat had no organization but depended on irregular raids.
So, faced with the Danish advance Alfred did not risk open battle but harried the enemy.
He was constantly on the move, drawing the Danes after him.
His patrols halted the raiding parties: hunger assailed the Danish army.
Now Alfred began a long series of skirmishes - and within a month the Danes had surrendered.


\item Silicon valley
What does the computer industry thrive on apart from anarchy? Technology trends may push Silicon Valley back to the future.
Carver Mead, a pioneer in integrated circuits and a professor of computer science at the California Institute of Technology, notes there are now workstations that enable engineers to design test and produce chips right on their desks, much the way an editor creates a newsletter on a Macintosh.
As the time and cost of making a chip drop to a few days and a few hundred dollars, engineers may soon be free to let their imaginations soar without being penalized by expensive failures.
Mead predicts that inventors will be able to perfect powerful customized chips over a weekend at the office - spawning a new generation of garage startups and giving the U.S. a jump on its foreign rivals in getting new products to market fast.
'Were got more garages with smart people', Mead observes.
'We really thrive on anarchy'.
And on Asians.
Already, orientals and Asian Americans constitute the majority of the engineering staffs at many Valley firms.
And Chinese Korean Filipino and Indian engineers are graduating in droves from Californias colleges.
As the heads of next-generation start-ups, these Asian innovators can draw on customs and languages to forge righter links with crucial Pacific Rim markets.
For instance, Alex Au, a Stanford Ph.D. from Hong Kong has set up a Taiwan factory to challenge Japans near lock on the memory-chip market.
India-born N.Damodar Reddy's tiny California company reopened an AT \&T chip plant in Kansas City last spring with financing from the state of Missouri.
Before it becomes a retirement village Silicon Valley may prove a classroom for building a global business.


\item  How to grow old


What according to the author is the best way to overcome the fear of death as you get older? Some old people are oppressed by the fear of death.
In the young there is a justification for this feeling.
Young men who have reason to fear that they will be killed in battle may justifiably feel bitter in the thought that they have been cheated of the best things that life has to offer.
But in an old man who has known human joys and sorrows, and has achieved whatever work it was in him to do, the fear of death is somewhat abject and ignoble.
The best way to overcome it - so at least it seems to me - is to make your interests gradually wider and more impersonal, until bit by bit the walls of the ego recede and your life becomes increasingly merged in the universal life.
An individual human existence should be like a river - small at first narrowly contained within its banks and rushing passionately past boulders and over waterfalls.
Gradually the river grows wider the banks recede the waters flow more quietly, and in the end, without any visible break they become merged in the sea and painlessly lose their individual being.
The man who, in old age, can see his life in this way will not suffer from the fear of death, since the things he cares for will continue.
And if with the decay of vitality, weariness increases, the thought of rest will be not unwelcome.
I should wish to die while still at work, knowing that others will carry on what I can no longer do, and content in the thought that what was possible has been done.

\item  Banks and their customers

Why is there no risk to the customer when a bank prints the customers name on his cheques? When anyone opens a current account at a bank, he is lending the bank money, repayment of which he may demand at any time, either in cash or by drawing a cheque in favor of another person.
Primarily the banker-customer relationship is that of debtor and creditor - who is which depending on whether the customers account is in credit or is overdrawn.
But in addition to that basically simple concept, the bank and its customer owe a large number of obligations to one another.
Many of these obligations can give in to problems and complications but a bank customer, unlike, say, a buyer of goods, cannot complain that the law is loaded against him.
The bank must obey its customer's instructions, and not those of anyone else.
When, for example a customer, first opens an account, he instructs the bank to debit his account only in respect of cheques draw by himself.
He gives the bank specimens of his signature, and there is a very firm rule that the bank has no right or authority to pay out a customer's money on a cheques on which its customer's signature has been forged.
It makes no difference that the forgery may have been a very skillful one, the bank must recognize its customer's signature.
For this reason there is no risk to the customer in the practice, adopted by banks of printing the customer's name on his cheques.
If this facilitates forgery, it is the bank which will lose, not the customer.


\item The search for oil

What do oilmen want to achieve as soon as they strike oil? The deepest holes of all made for oil and they go down to as much as 25,000 feet.
But we not need to send men down to get the oil out, as we must with other mineral deposits.
The holes are only borings, less than a foot in diameter.
My particular experience is largely in oil, and the search for oil has done more to improve deep drilling than any other mining activity.
When is has been decided where we are going to drill, we put up at the surface an oil derrick.
It has to be tall because it is like a giant block and tackle, and we have to lower into the ground and haul out of the ground great lengths of drill pipe which are rotated by an engine at the top and are fitted with a cutting bit at the bottom.
The geologist needs to know what rocks the drill has reached, so every so often a sample is obtained with a coring bit.
It cuts a clean cylinder of rock, from which can be seen the strata the drill has been cutting through.
Once we get down to the oil, it usually flows to the surface because great pressure, either from gas or water, is pushing it.
This pressure must be under control, and we control it by means of the mud which we circulate down the drill pipe.
We endeavor to avoid the old, romantic idea of a gusher, which wastes oil and gas.
We want it to stay down the hole until we can lead it off in a controlled manner.
                                                                                                                         
                                                                                                                          
\item The Butterfly Effect

Why do small errors make it impossible to predict the weather system with a high degree of accuracy? Beyond two or three days the worlds best weather forecasts are speculative, and beyond six or seven they are worthless.
The Butterfly Effect is the reason.
For small pieces of weather - and to a global forecaster small can mean thunderstorms and blizzards - any prediction deteriorates rapidly.
Errors and uncertainties multiply, cascading upward through a chain of turbulent features, from dust devils and squalls up to continent-size eddies that only satellites can see.
The modern weather models work with a grid of points of the order of sixty miles apart, and even so some starting data has to guessed, since ground stations and satellites cannot see everywhere.
But suppose the earth could be covered with sensors spaced one foot apart, rising at one-foot intervals all the way to the top of the atmosphere.
Suppose every sensor gives perfectly accurate readings of temperature, pressure, humidity, and any other quantity a meteorologist would want.
Precisely at noon an infinitely powerful computer takes all the data and calculates what will happen at each point at 12.01
then 12.02  then 12.03 ...
The computer will still be unable to predict whether Princeton, New Jersey, will have sun or rain on a day one month away.
At noon the spaces between the sensors will hide fluctuations that the computer will not know about, tiny deviations from the average.
By 12.01 those fluctuations will already have created small errors one foot away.
Soon the errors will have multiplied to the ten-foot scale, and so on up to the size of the globe.

\item Secrecy in industry

Why is secrecy particularly important in the chemical industries? Two factors weigh heavily against the effectiveness of scientific research in industry.
One is the general atmosphere of secrecy in which it is carried out, the other the lack of freedom of the individual research worker.
In so far as any inquiry is a secret one, it naturally limits all those engaged in carrying it out from effective contact with their fellow scientists either in other countries or in universities, or even, often enough, in other departments of the same firm.
The degree of secrecy naturally varies considerably.
Some of the bigger firms are engaged in researches which are of such general and fundamental nature that it is a positive advantage to them not to keep them secret.
Yet a great many processes depending on such research are sought for with complete secrecy until the stage at which patents can be taken out.
Even more processes are never patented at all but kept as secret processes.
This applies particularly to chemical industries, where chance discoveries play a much larger part than they do in physical and mechanical industries.
Sometimes the secrecy goes to such an extent that the whole nature of the research cannot be mentioned.
Many firms for instance have great difficulty in obtaining technical or scientific books from libraries because they are unwilling to have names entered as having taken out such and such a book for fear the agents of other firms should be able to trace the kind of research they are likely to be undertaking.


\item The modern city

What is the authors main argument about the modern city? In the organization of industrial life the influence of the factory upon the physiological and mental state of the workers has been completely neglected.
Modern industry is based on the conception of the maximum production at lowest cost, in order that an individual or a group of individuals may earn as much money as possible.
It has expanded without any idea of the true nature of the human beings who run the machines, and without giving any consideration to the effects produced on the individuals and on their descendants by the artificial mode of existence imposed by the factory.
The great cities have been built with no regard for us.
The shape and dimensions of the skyscrapers depend entirely on the necessity of obtaining the maximum income per square foot of ground, and of offering to the tenants offices and apartments that please them.
This caused the construction of gigantic buildings where too large masses of human beings are crowded together.
Civilized men like such a way of living.
While they enjoy the comfort and banal luxury of their dwelling, they do not realize that they are deprived of the necessities of life.
The modern city consists of monstrous edifices and of dark, narrow streets full of petrol fumes and toxic gases, torn by the noise of the taxicabs, lorries and buses, and thronged ceaselessly by great crowds.
Obviously it has not been planned for the good of its inhabitants.


\item         A manmade disease

What factor helped to spread the disease of myxomatosis? In the early days of the settlement of Australia enterprising settlers unwisely introduced the European rabbit.
This rabbit had no natural enemies in the Antipodes, so that it multiplied with that promiscuous abandon characteristic of rabbits.
It overran a whole continent.
It caused devastation by burrowing and by devouring the herbage which might have maintained millions of sheep and cattle.
Scientists discovered that this particular variety of rabbit (and apparently no other animal) was susceptible to a fatal virus disease myxomatosis.
By infecting animals and letting them loose in the burrows, local epidemics of this disease could be created.
Later it was found that there was a type of mosquito which acted as the carrier of this disease and passed it on to the rabbits.
So while the rest of the world was trying to get rid of mosquitoes Australia was encouraging this one.
It effectively spread the disease all over the continent and drastically reduced the rabbit population.
It later became apparent that rabbits were developing a degree of resistance to this disease, so that the rabbit population was unlikely to be completely exterminated.
There were hopes however that the problem of the rabbit would become manageable.
Ironically Europe which had bequeathed the rabbit as a pest to Australia acquired this manmade disease as a pestilence.
A French physician decided to get rid of the wild rabbits on his own estate and introduced myxomatosis.
It did not, however remain within the confines of his estate.
It spread through France Where wild rabbits are not generally regarded as a pest but as sport and a useful food supply, and it spread to Britain where wild rabbits are regarded as a pest but where domesticated rabbits, equally susceptible to the disease, are the basis of a profitable fur industry.
The question became one of whether Man could control the disease he had invented.

\item Porpoises

What would you say is the main characteristic of porpoises? There has long been a superstition among mariners that porpoises will save drowning men by pushing them to the surface, or protect them from sharks by surrounding them in defensive formation.
Marine Studio biologists have pointed out that, however intelligent they may be it is probably a mistake to credit dolphins with any motive of lifesaving.
On the occasions when they have pushed to shore an unconscious human being they have much more likely done it out of curiosity or for sport, as in riding the bow waves of a ship.
In  some porpoises were photographer working like beavers to push ashore a waterlogged mattress.
If as has been reported they have protected humans from sharks, it may have been because curiosity attracted them and because the scent of a possible meal attracted the sharks.
Porpoises and sharks are natural enemies.
It is possible that upon such an occasion a battle ensued, with the sharks being driven away or killed.
Whether it be bird, fish, or beast, the porpoise is intrigued with anything that is alive.
They are constantly after the turtles, who peacefully submit to all sorts of indignities.
One young calf especially enjoyed raising a turtle to the surface with his snout and then shoving him across the tank like an aquaplane.
Almost any day a young porpoise may be seen trying to turn a 300-pound sea turtle over by sticking his snout under the edge of his shell and pushing up for dear life.
This is not easy, and may require two porpoises working together.
In another game, as the turtle swims across the oceanarium, the first porpoise swoops down from above and butts his shell with his belly.
This knocks the turtle down several feet.
He no sooner recovers his equilibrium than the next porpoise comes along and hits him another crack.
Eventually the turtle has been butted all the way down to the floor of the tank.
He is now satisfied merely to try to stand up but as soon as he does so a porpoise knocks him flat.
The turtle at last gives up by pulling his feet under his shell and the game is over.

\item The stuff of dreams

What is going on when a person experiences rapid eyemovements during sleep? It is fairly clear that sleeping period must have some function, and because there is so much of it the function would seem to be important.
Speculations about is nature have been going on for literally thousands of years, and one odd finding that makes the problem puzzling is that it looks very much as if sleeping is not simply a matter of giving the body a rest.
'Rest', in terms of muscle relaxation and so on, can be achieved by a brief period lying, or even sitting down.
The body's tissues are self-repairing and self-restoring to a degree, and function best when more or less continuously active.
In fact a basic amount of movement occurs during sleep which is specifically concerned with preventing muscle inactivity.
If it is not a question of resting the body then perhaps it is the brain that needs resting? This might be a plausible hypothesis were it not for two factors.
First the electroencephalograph (which is simply a device for recording the electrical activity of the brain by attaching electrodes to the scalp) shows that while there is a change in the pattern of activity during sleep, there is no evidence that the total amount of activity is any less.
The second factor is more interesting and more fundamental.
Some years ago an American psychiatrist named William Dement published experiments dealing with the recording of eye-movements during sleep.
He showed that the average individual's sleep cycle is punctuated with peculiar bursts of eye-movements, some drifting and slow others jerky and rapid.
People woken during these periods of eye-movements generally reported that they had been dreaming.
When woken at other times they reported no dreams.
If one group of people were disturbed from their eye-movement sleep for several nights on end, and another group were disturbed for an equal period of time but when they were no exhibiting eye-movements, the first group began to show some personality disorders while the others seemed more or less unaffected.
The implications of all this were that it was not the disturbance of sleep that mattered, but the disturbance of dreaming.

\item  Snake poison

What are the two different ways in which snake poison acts? How it came about that snakes manufactured poison is a mystery.
Over the periods their saliva, a mild, digestive juice like our own, was converted into a poison that defies analysis even today.
It was not forced upon them by the survival competition; they could have caught and lived on prey without using poison, just as the thousands of nonpoisonous snakes still do.
Poison to a snake is merely a luxury; it enables it to get its food with very little effort, no more effort than one bite.
And why only snakes? Cats, for instance, would be greatly helped; no running fights with large, fierce rats or tussles with grown rabbits - just a bite and no more effort needed.
In fact, it would be an assistance to all carnivores though it would be a two-edged weapon when they fought each other.
But, of the vertebrates, unpredictable Nature selected only snakes (and one lizard).
One wonders also why Nature, with some snakes, concocted poison of such extreme potency.
In the conversion of saliva into poison, one might suppose that a fixed process took place.
It did not; some snakes manufacture a poison different in every respect from that of others, as different as arsenic is from strychnine, and having different effects.
One poison acts on the nerves, the other on the blood.
The makers of the nerve poison include the mambas and the cobras and their venom is called neurotoxic.
Vipers (adders) and rattlesnakes manufacture the blood poison, which is known as haemolytic.
Both poisons are unpleasant, but by far the more unpleasant is the blood poison.
It is said that the nerve poison is the more primitive of the two, that the blood poison is, so to speak, a newer product from an improved formula.
Be that as it may, the nerve poison does its business with man far more quickly than the blood poison.
This, however, means nothing.
Snakes did not acquire their poison for use against man but for use against prey such as rats and mice, and the effects on these of viperine poison is almost immediate.

\item        William S. Hart and

How did William Harts childhood prepare him for his acting role in Western films? William S. Hart was perhaps the greatest of all Western stars for unlike Gary Cooper and John Wayne he appeared in nothing but Westerns.
From  1914 to 1924 he was supreme and unchallenged.
It was Hart who created the basic formula of the Western film, and devised the protagonist he played in every film he made the good-bad man the accidental-noble outlaw or the honest-but-framed cowboy or the sheriff made suspect by vicious gossip; in short, the individual in conflict with himself and his frontier environment.
Unlike most of his contemporaries in Hollywood, Hart actually knew something of the old West.
He had lived in it as a child when it was already disappearing, and his hero was firmly rooted in his memories and experiences, and in both the history and the mythology of the vanished frontier.
And although no period or place in American history has been more absurdly romanticized, myth and reality did join hands in at least one arena, the conflict between the individual and encroaching civilization.
Men accustomed to struggling for survival against the elements and Indians were bewildered by politicians, bankers and businessmen, and unhorsed by fences, laws and alien taboos.
Harts good-bad man was always an outsider, always one of the disinherited, and if he found it necessary to shoot a sheriff or rob a bank along the way, his early audiences found it easy to understand and forgive, especially when it was Hart who, in the end, overcame the attacking Indians.
Audiences in the second decade of the twentieth century found it pleasant to escape to a time when life, though hard, was relatively simple.
We still do; living in a world in which undeclared aggression, war, hypocrisy, chicanery, anarchy and impending immolation are part of our daily lives, we all want a code to live by.

\item         Knowledge and progress
In what two areas have people made no progress at all? Why does the idea of progress loom so large in the modern world? Surely progress of a particular kind is actually taking place around us and is becoming more and more manifest.
 Although mankind has undergone no general improvement in intelligence or morality, it has made extraordinary progress in the accumulation of knowledge.
 Knowledge began to increase as soon as the thoughts of one individual could be communicated to another by means of speech.
 With the invention of writing, a great advance was made, for knowledge could then be not only communicated but also stored.
 Libraries made education possible, and education in its turn added to libraries: the growth of knowledge followed a kind of compound interest law, which was greatly enhanced by the invention of printing.
 All this was comparatively slow until, with the coming of science, the tempo was suddenly raised.
 Then knowledge began to be accumulated according to a systematic plan.
 The trickle became a stream; the stream has now become a torrent.
 Moreover, as soon as new knowledge is acquired, it is now turned to practical account.
 What is called 'modern civilization' is not the result of a balanced development of all man's nature, but of accumulated knowledge applied to practical life.
 The problem now facing humanity is What is going to be done with all this knowledge? As is so often pointed out knowledge is a two-edged weapon which can be used equally for good or evil.
 It is now being used indifferently for both.
 Could any spectacle, for instance, be more grimly whimsical than that of gunners using science to shatter men's bodies, while, close at hand, surgeons use it to restore them? We have to ask ourselves very seriously what will happen if this twofold use of knowledge with its ever-increasing power, continues.

\item        Bird flight

What are the two main types of bird flight described by the author? No two sorts of birds practise quite the same sort of flight; the varieties are infinite but two classes may be roughly seen.
Any ship that crosses the Pacific is accompanied for many days by the smaller albatross, which may keep company with the vessel for an hour without visible or more than occasional movement of wing.
The currents of air that the walls of the ship direct upwards, as well as in the line of its course, are enough to give the great bird with its immense wings sufficient sustenance and progress.
The albatross is the king of the gliders, the class of fliers which harness the air to their purpose, but must yield to its opposition.
In the contrary school, the duck is supreme.
It comes nearer to the engines with which man has 'conquered' the air, as he boasts.
Duck, and like them the pigeons, are endowed with steel-like muscles, that are a good part of the weight of the bird, and these will ply the short wings with such irresistible power that they can bore for long distances through an opposing gale before exhaustion follows.
Their humbler followers, such as partridges, have a like power of strong propulsion, but soon tire.
You may pick them up in utter exhaustion, if wind over the sea has driven them to a long journey.
The swallow shares the virtues of both schools in highest measure.
It tires not, nor does it boast of its power; but belongs to the air, travelling it may be six thousand miles to and from its northern nesting home, feeding its flown young as it flies, and slipping through we no longer take omens from their flight on this side and that; and even the most superstitious villagers no longer take off their hats to the magpie and wish it good-morning.

\item         Beatuy

What do glimpses of beauty either in nature or art often suggest to the human mind? A young man sees a sunset and, unable to understand or to express the emotion that it rouses in him, concludes that it must be the gateway to world that lies beyond.
It is difficult for any of us in moments of intense aesthetic experience to resist the suggestion that we are catching a glimpse of a light that shines down to us from a different realm of existence, different and, because the experience is intensely moving, in some way higher.
And, though the gleams blind and dazzle, yet do they convey a hint of beauty and serenity greater than we have known or imagined.
Greater too than we can describe; for language, which was invented to convey the meanings of this world, cannot readily be fitted to the uses of another.
That all great has this power of suggesting a world beyond is undeniable.
In some moods, Nature shares it.
There is no sky in June so blue that it does not point forward to a bluer, no sunset so beautiful that it does not waken the vision of a greater beauty, a vision which passes before it is fully glimpsed, and in passing leaves and indefinable longing and regret.
But, if this world is not merely a bad joke, life a vulgar flare amid the cool radiance of the stars, and existence an empty laugh braying across the mysteries; if these intimations of a something behind and beyond are not evil humour born of indigestion, or whimsies sent by the devil to mock and madden us, if, in a word, beauty means something, yet we must not seek to interpret the meaning.
If we glimpse the unutterable, it is unwise to try to utter it, nor should we seek to invest with significance that which we cannot grasp.
Beauty in terms of our human meanings is meaningless.

\item        Non-auditory effects of noise

What conclusion does the author draw about noise and health in this piece? May people in industry and the Services, who have practical experience of noise, regard any investigation of this question as a waste of time; they are not prepared even to admit the possibility that noise affects people.
 On the other hand, those who dislike noise will sometimes use most inadequate evidence to support their pleas for a quieter society.
 This is a pity. because noise abatement really is a good cause, and it is likely to be discredited if it gets to be associated with had science.
One allegation often made is that noise produces mental illness.
 A recent article in a weekly newspaper, for instance, was headed with a striking illustration of a lady in a state of considerable distress, with the caption 'She was yet another victim, reduced to a screaming wreck'.
 On turning eagerly to the text, one learns that the lady was a typist who found the sound of office typewriters worried her more and more until eventually she had to go into a mental hospital.
 Now the snag in this sort of anecdote is of course that one cannot distinguish cause and effect. Was the noise a cause of the illness, or were the complaints about noise merely a symptom? Another patient might equally well complain that her neighbours were combining to slander her and persecute her, and yet one might be cautious about believing this statement.
What is needed in case of noise is a study of large numbers of people living under noisy conditions, to discover whether they are mentally ill more often than other people are.
 Some time ago the United States Navy, for instance, examined a very large number of men working on aircraft carriers: the study was known as Project Anehin.
 It can be unpleasant to live even several miles from an aerodrome; if you think what it must be like to share the deck of a ship with several squadrons of jet aircraft, you will realize that a modern navy is a good place to study noise.
 But neither psychiatric interviews nor objective tests were able to show any effects upon these American sailors.
 This result merely confirms earlier American and British studies: if there is any effect of noise upon mental health, it must be so small that present methods of psychiatric diagnosis cannot find it.
 That does not prove that it does exist; but it does mean that noise is less dangerous than say being brought up in an orphanage -   which really is a mental health hazard.

\end{enumerate}


